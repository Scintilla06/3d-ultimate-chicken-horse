%%
%% This is file `sample-sigconf-authordraft.tex',
%% generated with the docstrip utility.
%%
%% The original source files were:
%%
%% samples.dtx  (with options: `all,proceedings,bibtex,authordraft')
%% 
%% IMPORTANT NOTICE:
%% 
%% For the copyright see the source file.
%% 
%% Any modified versions of this file must be renamed
%% with new filenames distinct from sample-sigconf-authordraft.tex.
%% 
%% For distribution of the original source see the terms
%% for copying and modification in the file samples.dtx.
%% 
%% This generated file may be distributed as long as the
%% original source files, as listed above, are part of the
%% same distribution. (The sources need not necessarily be
%% in the same archive or directory.)
%%
%%
%% Commands for TeXCount
%TC:macro \cite [option:text,text]
%TC:macro \citep [option:text,text]
%TC:macro \citet [option:text,text]
%TC:envir table 0 1
%TC:envir table* 0 1
%TC:envir tabular [ignore] word
%TC:envir displaymath 0 word
%TC:envir math 0 word
%TC:envir comment 0 0
%%
%% The first command in your LaTeX source must be the \documentclass
%% command.
%%
%% For submission and review of your manuscript please change the
%% command to \documentclass[manuscript, screen, review]{acmart}.
%%
%% When submitting camera ready or to TAPS, please change the command
%% to \documentclass[sigconf]{acmart} or whichever template is required
%% for your publication.
%%
%%
\documentclass[sigconf]{acmart}
\settopmatter{printacmref=false}
\setcopyright{none}
\renewcommand\footnotetextcopyrightpermission[1]{}
%%
%% \BibTeX command to typeset BibTeX logo in the docs
\AtBeginDocument{%
  \providecommand\BibTeX{{%
    Bib\TeX}}}

%% Rights management information.  This information is sent to you
%% when you complete the rights form.  These commands have SAMPLE
%% values in them; it is your responsibility as an author to replace
%% the commands and values with those provided to you when you
%% complete the rights form.
\setcopyright{acmlicensed}
\copyrightyear{2018}
\acmYear{2018}
\acmDOI{XXXXXXX.XXXXXXX}
%% These commands are for a PROCEEDINGS abstract or paper.
\acmConference[Conference acronym 'XX]{Make sure to enter the correct
  conference title from your rights confirmation email}{June 03--05,
  2018}{Woodstock, NY}
%%
%%  Uncomment \acmBooktitle if the title of the proceedings is different
%%  from ``Proceedings of ...''!
%%
%%\acmBooktitle{Woodstock '18: ACM Symposium on Neural Gaze Detection,
%%  June 03--05, 2018, Woodstock, NY}
\acmISBN{978-1-4503-XXXX-X/2018/06}


%%
%% Submission ID.
%% Use this when submitting an article to a sponsored event. You'll
%% receive a unique submission ID from the organizers
%% of the event, and this ID should be used as the parameter to this command.
%%\acmSubmissionID{123-A56-BU3}

%%
%% For managing citations, it is recommended to use bibliography
%% files in BibTeX format.
%%
%% You can then either use BibTeX with the ACM-Reference-Format style,
%% or BibLaTeX with the acmnumeric or acmauthoryear sytles, that include
%% support for advanced citation of software artefact from the
%% biblatex-software package, also separately available on CTAN.
%%
%% Look at the sample-*-biblatex.tex files for templates showcasing
%% the biblatex styles.
%%

%%
%% The majority of ACM publications use numbered citations and
%% references.  The command \citestyle{authoryear} switches to the
%% "author year" style.
%%
%% If you are preparing content for an event
%% sponsored by ACM SIGGRAPH, you must use the "author year" style of
%% citations and references.
%% Uncommenting
%% the next command will enable that style.
%%\citestyle{acmauthoryear}


%%
%% end of the preamble, start of the body of the document source.
\begin{document}

%%
%% The "title" command has an optional parameter,
%% allowing the author to define a "short title" to be used in page headers.
\title{Mid-term Written Report: Ultimate Chicken Horse 3D}
\subtitle{Topic 2: Interactive Game}

%%
%% The "author" command and its associated commands are used to define
%% the authors and their affiliations.
%% Of note is the shared affiliation of the first two authors, and the
%% "authornote" and "authornotemark" commands
%% used to denote shared contribution to the research.
\author{An Yan}
\email{yana24@mails.tsinghua.edu.cn}
\affiliation{%
  \institution{Tsinghua University}
  \city{Beijing}
  \country{China}
}

\author{Zhenbang Pan}
\email{pzb24@mails.tsinghua.edu.cn}
\affiliation{%
  \institution{Tsinghua University}
  \city{Beijing}
  \country{China}
}

%%
%% By default, the full list of authors will be used in the page
%% headers. Often, this list is too long, and will overlap
%% other information printed in the page headers. This command allows
%% the author to define a more concise list
%% of authors' names for this purpose.
\renewcommand{\shortauthors}{Yan and Pan}

%%
%% The abstract is a short summary of the work to be presented in the
%% article.
\begin{abstract}
This report summarizes the progress of our project "Ultimate Chicken Horse 3D", a web-based multiplayer party platformer game. We detail the project goals, technical points, finished aspects including the toon shading rendering pipeline and physics integration, and the plan for remaining tasks such as networking and character animation.
\end{abstract}

%%
%% The code below is generated by the tool at http://dl.acm.org/ccs.cfm.
%% Please copy and paste the code instead of the example below.
%%
% \begin{CCSXML}
% <ccs2012>
%  <concept>
%   <concept_id>00000000.0000000.0000000</concept_id>
%   <concept_desc>Do Not Use This Code, Generate the Correct Terms for Your Paper</concept_desc>
%   <concept_significance>500</concept_significance>
%  </concept>
%  <concept>
%   <concept_id>00000000.00000000.00000000</concept_id>
%   <concept_desc>Do Not Use This Code, Generate the Correct Terms for Your Paper</concept_desc>
%   <concept_significance>300</concept_significance>
%  </concept>
%  <concept>
%   <concept_id>00000000.00000000.00000000</concept_id>
%   <concept_desc>Do Not Use This Code, Generate the Correct Terms for Your Paper</concept_desc>
%   <concept_significance>100</concept_significance>
%  </concept>
%  <concept>
%   <concept_id>00000000.00000000.00000000</concept_id>
%   <concept_desc>Do Not Use This Code, Generate the Correct Terms for Your Paper</concept_desc>
%   <concept_significance>100</concept_significance>
%  </concept>
% </ccs2012>
% \end{CCSXML}

% \ccsdesc[500]{Do Not Use This Code~Generate the Correct Terms for Your Paper}
% \ccsdesc[300]{Do Not Use This Code~Generate the Correct Terms for Your Paper}
% \ccsdesc{Do Not Use This Code~Generate the Correct Terms for Your Paper}
% \ccsdesc[100]{Do Not Use This Code~Generate the Correct Terms for Your Paper}

%%
%% Keywords. The author(s) should pick words that accurately describe
%% the work being presented. Separate the keywords with commas.
% \keywords{Do, Not, Use, This, Code, Put, the, Correct, Terms, for,
%   Your, Paper}
%% A "teaser" image appears between the author and affiliation
%% information and the body of the document, and typically spans the
%% page.
% \begin{teaserfigure}
%   \includegraphics[width=\textwidth]{sampleteaser}
%   \caption{Seattle Mariners at Spring Training, 2010.}
%   \Description{Enjoying the baseball game from the third-base
%   seats. Ichiro Suzuki preparing to bat.}
%   \label{fig:teaser}
% \end{teaserfigure}

\received{20 February 2007}
\received[revised]{12 March 2009}
\received[accepted]{5 June 2009}

%%
%% This command processes the author and affiliation and title
%% information and builds the first part of the formatted document.
\maketitle

\section{Project Goal}
We aim to develop a web-based, 3D multiplayer party platformer game inspired by \textit{Ultimate Chicken Horse}. The core gameplay loop involves two distinct phases: a \textbf{Building Phase}, where players strategically place traps and platforms to hinder opponents, and a \textbf{Running Phase}, where players compete to reach the goal using the built environment. The game features a stylized cartoon aesthetic and utilizes a P2P network architecture for online multiplayer interaction.


\section{Technical Points}
Beyond the basic requirements , we plan to implement the following advanced features:

\begin{itemize}
    \item \textbf{An Yan:}
    \begin{itemize}
        \item \textbf{Online multiplayer system (4pts):} Implementing a Host-Authoritative P2P architecture using WebRTC.
        \item \textbf{Customizable character appearance (2pts):} Allowing players to select different animal skins using texture atlas UV mapping.
        \item \textbf{Easy installation or online access (2pts):} Deploying via GitHub Pages with automated CI/CD.
        \item \textbf{Environment lighting (1pts):} Implementing Toon-style lighting with rim lights and shadow maps.
    \end{itemize}
    \item \textbf{Zhenbang Pan:}
    \begin{itemize}
        \item \textbf{Articulated animals with rigging and skinning \,(3 pts):} Creating a repeatable rigging/skin workflow for animal characters with blender, Mixamo and Three.js.
        \item \textbf{Surface interaction physics \,(2 pts):} Implementing gameplay surfaces with distinct contact properties, e.g. honey/ice.
        \item \textbf{Additional auxiliary interfaces in 3D \,(2 pts):} Designing in-world 3D roll/tool selection interfaces highlighting intuitiveness and interactivity.
        \item \textbf{User-friendly layout with visually appealing design \,(1 pt):} Implementing UI components embedded in 3D that provide consistent style, immersive experience and smooth camera transition.
        \item \textbf{Synchronized audio \,(1 pt):} Event-driven audio system (BGM + SFX) with positional audio for key events (e.g. trap trigger, jump, goal).
    \end{itemize}
\end{itemize}

\section{External Tools}
\begin{itemize}
    \item \textbf{Rendering \& Engine:} Three.js (WebGL), Vite (Build Tool).
    \item \textbf{Physics Simulation:} Cannon-es (Rigid body physics).
    \item \textbf{Networking:} PeerJS (WebRTC wrapper for P2P connection).
    \item \textbf{Modeling \& Animation:} Nano Banana Pro (text-to-image generation) for concept images, and the image-to-3D pipeline from Xiang et al.\cite{xiang2025structured3dlatentsscalable} to convert those images into 3D models; Blender (cleanup, UV mapping, harden normals) for final asset preparation, and Mixamo (auto-rigging) when appropriate.
    \item \textbf{Deployment:} GitHub Actions (CI/CD).
\end{itemize}

\section{Finished Technical Aspects}
We have established the core technical framework and art pipeline.

\begin{itemize}
    \item \textbf{Rendering Pipeline (Cartoon Style):}
    \begin{itemize}
        \item We rejected standard PBR rendering in favor of a stylized \textbf{Toon Shader}. We implemented \texttt{MeshToonMaterial} with a custom 16-step Gradient Map to achieve distinct light banding.
        \item Resolved "pillow shading" artifacts on low-poly models by using \textbf{Harden Normals} in Blender and correct export settings.
    \end{itemize}
    \item \textbf{Asset Pipeline:}
    \begin{itemize}
        \item Models for gameplay tools (platforms, spikes, crossbows) and character roster (chicken, horse, sheep, monkey, raccoon, lizard, rabbit) were produced using the modeling pipeline described in the "External Tools" section (Nano Banana Pro → Structured 3D Latents image→3D pipeline → Blender cleanup/UV → optional Mixamo rigging).
    \end{itemize}
    \item \textbf{Physics Engine Integration \& Gameplay Items:}
    \begin{itemize}
        \item Integrated \textbf{Cannon-es} with Three.js. Implemented a sync mechanism where the physics body drives the visual mesh.
        \item Implemented diverse gameplay items with specific physics logic, focusing on \textbf{Traps}: Implemented \textit{Crossbows} with automatic firing timers and object-pooled projectiles, and \textit{Spikes} as static hazards.
    \end{itemize}
    \item \textbf{3D in-world UI:}
    \begin{itemize}
        \item Implemented UI components as \textbf{Three.js Scene Objects} (menu, roll selection, tool selection).
        \item Implemented \textbf{Smooth Camera Transition} to shift the player’s focus naturally and immersively between in-world UI elements, such as menus, tool selection, and the build/run mode.
    \end{itemize}
    \item \textbf{Networking Infrastructure:}
    \begin{itemize}
        \item Established the core P2P communication architecture using \textbf{PeerJS}.
        \item Implemented connection lifecycle management (host/client handshake) and a packet-based communication protocol supporting both unicast and broadcast.
    \end{itemize}
    \item \textbf{Infrastructure:}
    \begin{itemize}
        \item Set up the \textbf{GitHub Actions} workflow. The project is automatically built and deployed to \texttt{github.io} upon pushing to the main branch, solving path resolving issues (base URL configuration).
    \end{itemize}
\end{itemize}

% Placeholder for visual results
\begin{figure}[h]
  \centering
  \includegraphics[width=\linewidth]{image.png}
  \caption{Current game scene}
  \Description{A screenshot of the game showing the toon shading style and outlines.}
  \label{fig:game_scene}
\end{figure}

\section{Plan for Remaining Technical Tasks}
\begin{itemize}
    \item \textbf{Networking (Gameplay Sync):} Building upon the established P2P infrastructure, finalize the state synchronization for the "Building Phase" (syncing object placement) and "Running Phase" (interpolating player positions to handle latency).
    \item \textbf{Character Animation:} Rig and animate models produced by the External Tools pipeline (Mixamo or manual rigging), and implement the \texttt{AnimationMixer} state machine to blend Idle, Run and Jump.
    \item \textbf{Interactive Items \& Surfaces:} Implement physics logic for "Honey" (high friction) and "Ice" (zero friction). Implement interactive objects including \textit{Springs} (high restitution), \textit{Conveyors} (constant force), and \textit{Coins} (collection logic).
    \item \textbf{Game Loop Logic:} Implement the turn-based logic, scoring system, and the "Replay System" (recording state snapshots for playback).
    \item \textbf{Audio \& Polish:} Add spatial sound effects and refine the UI interactions.
\end{itemize}

\section{Detailed Schedule}
\begin{itemize}
    \item \textbf{Nov 15 - Nov 30:}
    \begin{itemize}
        \item Project initialization, tech stack selection (Three.js + Cannon-es), and setting up the CI/CD pipeline. (Completed)
        \item Building up core gameplay prototype, implementing the "Building Phase" (Raycasting for object placement) and basic "Running Phase" physics. (Completed)
    \end{itemize}
    \item \textbf{Dec 1 - Dec 14:}
    \begin{itemize}
        \item Implementing in-world 3D UI and art style validation. (Completed)
        \item Implementing PeerJS data channels for syncing game states and player inputs.
        \item Importing rigged characters, setting up the animation state machine, and implementing character customization.
    \end{itemize}
    \item \textbf{Dec 15 - Dec 31:}
    \begin{itemize}
        \item Gameplay refinement, adding interactive items (Springs, Conveyors, Coins) and fluid/ice mechanics.
        \item Implementing audio, scoring logic, and the replay system.
        \item UI beautification, bug fixing, and preparation for the In-class Presentation.
    \end{itemize}
    \item \textbf{Jan 1 - Jan 14:}
    \begin{itemize}
        \item Final Report writing and code cleanup.
    \end{itemize}
\end{itemize}
\bibliographystyle{ACM-Reference-Format}
\bibliography{references}
\end{document}
\endinput
%%
%% End of file `sample-sigconf-authordraft.tex'.